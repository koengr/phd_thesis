
%
%


First and foremost, I would like to thank my advisors Harry and Kareljan. As a starting PhD student, I came out of a previous project in artificial intelligence and felt like I had very limited background knowledge about the field of quantum computing, neither about the condensed matter theory required to understand small quantum computers, nor the advanced mathematics needed to understand the computer science aspects and algorithms. Nevertheless, Harry and Kareljan remained resolutely positive and enthusiastic about my work, even when our first results were not received as well as we had hoped. 
%
Please keep doing this for all your future students, especially to those that had some setbacks, as I believe that a student that feels that he/she is doing well can be much more motivated. 

Especially to Kareljan, I greatly enjoyed our collaboration on the papers on multiqubit gates, which resulted in \cref{part:multiqubit} of this thesis. I learned a lot about condensed matter techniques, writing good papers, and dealing with the politics that pop up when dealing with journals or other scientists. 
Apart from the papers we wrote together, you were also a great help with my other results on state transfer. I ended up appealing to you for various technical questions, for advise about presentation of the result, about potential journals, and for proof-reading the manuscripts. In the end, you suggested that you did not need to be part of the list of authors - a very generous gesture that endowed me the scientific honor of a single-author paper. Many thanks for all of the above!

Having made my first steps into the world of state transfer, I appealed to Carla and Reinier to help me deal with a question about eigenvalues of graphs. Initially, I was hoping that the answer to this question was already known for decades in the mathematics community, providing me some lemma's that I could readily turn into a physics paper. Of course, nothing comes as easy as that, and we ended up solving this problem by ourselves. I greatly enjoyed this collaboration, blending our different expertises, which I think yielded a great result. 

I am also grateful for much help that I got on the contents on this thesis from Jasper and Rene. The state transfer ideas all incubated thanks to Jasper's lectures I was allowed to visit, and found a more concrete form thanks to many later discussions. Rene's expertise on trapped ions proved invaluable to develop the ideas on the Ising star model in an experimental context. 

To the reading committee of this thesis: many thanks for taking your time to evaluate my dissertation, and to take part in the opposition during the defense. 


To all the fellow junior researchers at and around QuSoft, including Freek, Joris, Tom, Joran, \'{A}lvaro, Farrokh, Arjan, Yfke, Florian, Jan, Jana, Lars, Srini, Teresa, Simon, Jonas, Sebastian, Bas, Harold, Chris Majenz, Chris Cade, Mathys, Ralph, Andr\'{a}s, Sander, Isabella, Ruben, Ido, Alex, Yinan, Subha: thanks for the wonderful time at CWI. I will surely remember all the blackboard discussions, the intense fu{\ss}ball matches, the lunches, the conferences, and the occasional cakes. To Freek and Joris, many thanks for all the aid with more involved mathematics, especially the representation theory of spins. Especially to Tom, it was great to start our PhD programmes (pretty much) at the same time and to grow to become increasingly experienced scientists together. I am also indebted to you for all the computer help you gave me, including simple stuff such as how to exit VIM, but also running computations on parallel machines and hacking the screen resolution of old displays. %

%

\paragraph{}
De afgelopen vier jaar waren natuurlijk ook buiten werk om een leuke tijd, vooral dankzij alle mensen die ik heb ontmoet of beter heb leren kennen in deze periode. 
%

Bedankt aan mijn huisgenoten, voor gezellige etentjes, de borrels en feestjes, en alle discussies over het leven, de maatschappij, of natuurkunde (of het tolereren ervan). 

Aan mijn vrienden in de schaduwhark, dankzij jullie ben ik dusdanig enthousiast geworden over natuurkunde dat ik er zelfs mijn werk van heb gemaakt. Bedankt voor alle jarenlange gezelligheid, de festivals, de reisjes voor wintersport of surfen, en natuurlijk de legendarische pannenkoeken-avonden. 

Voor alle natuur- en wiskunden bij NSA, bedankt dat jullie zorgen dat de UvA niet alleen een geweldige plek is om dingen te leren, maar ook een uitstekende bron van gezelligheid en een locatie voor de gekste evenementen. 

Aan de volleyballers van UvO en daarbuiten, bedankt voor alle geweldige momenten in de sportzaal, op het zand, op het gras, of in de kroeg.  

Bart, Pelle, Nik\`{e}, Alexander en Manon, geweldig dat we elkaar nogsteeds regelmatig zien; ik geniet altijd erg van onze spelletjesmiddagen, etentjes, en natuurlijk de decenniale reis naar Rome. 

Aan mijn paranimfen Etienne en Leonie: bedankt dat jullie me willen versterken bij de organisatie van alles wat bij de promotie komt kijken.

Etienne, over de afgelopen tien jaar heb ik gigantisch veel van je geleerd, over uiteenlopende onderwerpen zoals lifehacks, wetenschap, sport (van triathlons tot Cleveland Browns en Ajax) en heel veel muziek. 

Lieve Leonie, het was geweldig om elkaar deze jaren steeds beter te leren kennen. Ik ben dol op ons leven vol reizen, sporten, festivals en andere activiteiten, en ik hoop dat we na onze PhDs hier ook nog vele jaren mee doorgaan. 

Aan mijn ouders Jos en Annelies, en zusje Carla, ontzettend bedankt dat jullie er altijd voor me zijn, voor de goede zorgen, voor al het belangrijke levensadvies, en dat ik bij jullie altijd een veilige plek heb waar ik terug kan keren. 

