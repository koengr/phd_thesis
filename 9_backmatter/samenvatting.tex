De belangrijkste resultaten van dit proefschrift verdelen we in twee categorie\"{e}n. In Deel \ref{part:multiqubit} behandelen we protocollen die leiden tot logische poorten (\emph{gates}) op meeredere qubits door middel van resonante aansturing (\emph{driving}), en in Deel \ref{part:adiabatic} richten we ons op het adiabatisch verplaatsen van quantumtoestanden. De voorgaande hoofdstukken in Deel \ref{part:preliminaries} bevatten een introductie tot deze onderwerpen. 

\paragraph{} In \textbf{Deel II} construeren we multiqubit gates, zoals de \texttt{Toffoli}$_N$. Deze zijn essentieel in veel quantumalgoritmes, maar blijken lastig uit te voeren op experimentele quantumcomputers. In tegenstelling tot de gebruikelijke aanpak waarbij grote operaties worden gecompileerd to een reeks schakelingen op weinig qubits, gebruiken wij een Hamiltoniaan de leidt tot een continue evolutie in een sterk gekoppeld quantumsysteem. De selectiviteit van resonant driving maakt het mogelijk om slechts $2$ van de $2^N$ eigentoestanden van een achtergrondveld te verwisselen, wat leidt tot een operatie die we \texttt{iSWAP} noemen. Om deze operatie nuttig te maken voor quantumcomputers, introduceren we \emph{eigengates} die deze eigentoestanden terugzetten naar de computationele basis. 

We bouwen en simuleren dergelijke gates in drie verschillende systemen. De eerste is de Krawtchoukketen, een voorbeeld van een XX keten met enkel koppelingen tussen naastgelegen qubits ($H = \sum_{j} J_j \left( X_j X_{j+1} + Y_j Y_{j+1} \right)$). Het systeem heeft bijzondere eigenschappen, zoals een lineaire dispersie, een tijdsevolutie die elke toestand doet spiegelen, en bovendien een overzichtelijke eigengate. We vinden dat lokale driving kan leiden tot unieke overgangen tussen de toestanden $\ket{1^\frac{N}{2} ~ 0^\frac{N}{2}}$ en $\ket{0^\frac{N}{2} ~ 1^\frac{N}{2}}$, en we analyseren numeriek de fouten die optreden bij ketens van lengte $N=4$ en $N=6$. 

\sloppy{Ten tweede beschouwen we een voorbeeld van een XXX keten met koppelingen over lange afstanden ($H = \sum_{j<k} J_{j,k} \left( \vec{\sigma}_j \cdot \vec{\sigma}_k \right)$), welke door Polychronakos is ge\"{i}ntroduceerd als een variant op het Haldane-Shastry model. Een quench met deze Hamiltoniaan blijkt zelf een eigengate te zijn, en wel voor een keten van de vorm $H = \sum_{j<k} J_{jk} \left( X_j Y_k - Y_j X_k \right)$. Weer simuleren we de fouten die optreden bij resonant driving voor lengtes $N=4$ en $N=6$. }

De vorige twee systemen hebben helaas kleine matrixelementen tussen de eigentoestanden die we willen verwisselen, waardoor de gates lang duren. Daarom bekijken we een derde model die dit probleem niet heeft, namelijk een Ising model ($H = \sum_{jk} J_{jk} Z_j Z_k$) waarbij een speciale qubit een positieve koppeling heeft met elk ander qubit.  De speciale qubit maakt onder resonant driving enkel een overgang wanneer de andere qubits de toestand $\ket{1}$ aannemen, precies zoals bij de \texttt{Toffoli}-gate. We vinden dat de benodigde tijd voor onze gate dit keer constant blijft onder toenemende $N$, wat een verbetering is over gebruikelijke compileertechnieken. Echter, onze aannames over het koppelen van het speciale qubit met $N$ anderen, en de driving frequentie die steeds groter wordt, zijn mogelijk niet realistisch voor grote $N$. Toch kan dit protocol nuttig zijn voor middelgrote quantumcomputers.


\paragraph{} 
In \textbf{Deel III} bestuderen we protocollen waarbij een quantumtoestand wordt verplaatst door een natuurkundig model gedefinieerd op een graaf. Dergelijke protocollen worden veelvuldig toegepast bij huidige natuur- en scheikunde-experimenten, en worden mogelijk belangrijk voor toekomstige quantumcomputers. We beschouwen twee verschillende modellen.

In ons eerste model kan een enkel kwantumdeeltje tunnelen tussen de knopen van een graaf. Twee bestaande protocollen, met de namen STImulated Raman Adiabatic Passage (STIRAP) and Coherent Tunnelling by Adiabatic Passage (CTAP), staan toe zulke deeltjes te verplaatsen tussen de eindpunten van een keten, waarbij het systeem telkens precies nul energie heeft. Ons resultaat is dat deze protocollen op veel meer grafen werken, namelijk (semi-)bipartite grafen met een volmaakte knopenkoppeling wanneer de zender wordt weggelaten, en wanneer de ontvanger wordt weggelaten. Veel van de belangrijke voordelen van STIRAP en CTAP, zoals bescherming tegen bepaalde vormen van decoherentie, blijven gelden. We testen het protocol op een boomstructuur, waarbij de zender en ontvanger zich op een blad bevinden. Het protocol blijkt verrassend accuraat, vooral wanneer de verzender en ontvanger een zwakkere interactie hebben dan de overige knopen. 

Het tweede model is een anti-ferromagnetisch XXX spinsysteem, waarbij de interacties tussen spindeeltjes worden vastgelegd door een graaf. We vinden weer dat eerdere protocollen op lineaire ketens ook toepasbaar zijn op algemenere grafen, namelijk bipartite grafen met een bepaalde balans tussen de maximale spin op de partities. Onder vergelijkbare voorwaarden kan verstrengeling tussen verweggelegen knopen worden gevormd. We analyseren de nauwkeurigheid van de protocollen op een kleine stervormige graaf, en bespreken een mogelijke experimentele test. 
